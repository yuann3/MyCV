\documentclass[10pt, letterpaper]{article}

% Packages:
\usepackage[
    ignoreheadfoot, % set margins without considering header and footer
    top=2 cm, % seperation between body and page edge from the top
    bottom=2 cm, % seperation between body and page edge from the bottom
    left=2 cm, % seperation between body and page edge from the left
    right=2 cm, % seperation between body and page edge from the right
    footskip=1.0 cm, % seperation between body and footer
    % showframe % for debugging 
]{geometry} % for adjusting page geometry
\usepackage{titlesec} % for customizing section titles
\usepackage{tabularx} % for making tables with fixed width columns
\usepackage{array} % tabularx requires this
\usepackage[dvipsnames]{xcolor} % for coloring text
\definecolor{primaryColor}{RGB}{0, 0, 0} % define primary color
\usepackage{enumitem} % for customizing lists
\usepackage{fontawesome5} % for using icons
\usepackage{amsmath} % for math
\usepackage[
    pdftitle={Yiyuan Li's CV},
    pdfauthor={Yiyuan Li},
    pdfcreator={LaTeX CV Yiyuan},
    colorlinks=true,
    urlcolor=primaryColor
]{hyperref} % for links, metadata and bookmarks
\usepackage[pscoord]{eso-pic} % for floating text on the page
\usepackage{calc} % for calculating lengths
\usepackage{bookmark} % for bookmarks
\usepackage{lastpage} % for getting the total number of pages
\usepackage{changepage} % for one column entries (adjustwidth environment)
\usepackage{paracol} % for two and three column entries
\usepackage{ifthen} % for conditional statements
\usepackage{needspace} % for avoiding page brake right after the section title
\usepackage{iftex} % check if engine is pdflatex, xetex or luatex

% Ensure that generate pdf is machine readable/ATS parsable:
\ifPDFTeX
    \input{glyphtounicode}
    \pdfgentounicode=1
    \usepackage[T1]{fontenc}
    \usepackage[utf8]{inputenc}
    \usepackage{lmodern}
\fi

\usepackage{charter}

% Some settings:
\raggedright
\AtBeginEnvironment{adjustwidth}{\partopsep0pt} % remove space before adjustwidth environment
\pagestyle{empty} % no header or footer
\setcounter{secnumdepth}{0} % no section numbering
\setlength{\parindent}{0pt} % no indentation
\setlength{\topskip}{0pt} % no top skip
\setlength{\columnsep}{0.15cm} % set column seperation
\pagenumbering{gobble} % no page numbering

\titleformat{\section}{\needspace{4\baselineskip}\bfseries\large}{}{0pt}{}[\vspace{1pt}\titlerule]

\titlespacing{\section}{
    % left space:
    -1pt
}{
    % top space:
    0.3 cm
}{
    % bottom space:
    0.2 cm
} % section title spacing

\renewcommand\labelitemi{$\vcenter{\hbox{\small$\bullet$}}$} % custom bullet points
\newenvironment{highlights}{
    \begin{itemize}[
        topsep=0.10 cm,
        parsep=0.10 cm,
        partopsep=0pt,
        itemsep=0pt,
        leftmargin=0 cm + 10pt
    ]
}{
    \end{itemize}
} % new environment for highlights


\newenvironment{highlightsforbulletentries}{
    \begin{itemize}[
        topsep=0.10 cm,
        parsep=0.10 cm,
        partopsep=0pt,
        itemsep=0pt,
        leftmargin=10pt
    ]
}{
    \end{itemize}
} % new environment for highlights for bullet entries

\newenvironment{onecolentry}{
    \begin{adjustwidth}{
        0 cm + 0.00001 cm
    }{
        0 cm + 0.00001 cm
    }
}{
    \end{adjustwidth}
} % new environment for one column entries

\newenvironment{twocolentry}[2][]{
    \onecolentry
    \def\secondColumn{#2}
    \setcolumnwidth{\fill, 4.5 cm}
    \begin{paracol}{2}
}{
    \switchcolumn \raggedleft \secondColumn
    \end{paracol}
    \endonecolentry
} % new environment for two column entries

\newenvironment{threecolentry}[3][]{
    \onecolentry
    \def\thirdColumn{#3}
    \setcolumnwidth{, \fill, 4.5 cm}
    \begin{paracol}{3}
    {\raggedright #2} \switchcolumn
}{
    \switchcolumn \raggedleft \thirdColumn
    \end{paracol}
    \endonecolentry
} % new environment for three column entries

\newenvironment{header}{
    \setlength{\topsep}{0pt}\par\kern\topsep\centering\linespread{1.5}
}{
    \par\kern\topsep
} % new environment for the header

\newcommand{\placelastupdatedtext}{% \placetextbox{<horizontal pos>}{<vertical pos>}{<stuff>}
  \AddToShipoutPictureFG*{% Add <stuff> to current page foreground
    \put(
        \LenToUnit{\paperwidth-2 cm-0 cm+0.05cm},
        \LenToUnit{\paperheight-1.0 cm}
    ){\vtop{{\null}\makebox[0pt][c]{
        \small\color{gray}\textit{Last updated in September 2024}\hspace{\widthof{Last updated in September 2024}}
    }}}%
  }%
}%

% save the original href command in a new command:
\let\hrefWithoutArrow\href

% new command for external links:


\begin{document}
    \newcommand{\AND}{\unskip
        \cleaders\copy\ANDbox\hskip\wd\ANDbox
        \ignorespaces
    }
    \newsavebox\ANDbox
    \sbox\ANDbox{$|$}

    \begin{header}
        \fontsize{25 pt}{25 pt}\selectfont Yiyuan Li

        \vspace{5 pt}

        \normalsize
        \mbox{Singapore,  River Vally}%
        \kern 5.0 pt%
        \AND%
        \kern 5.0 pt%
        \mbox{\hrefWithoutArrow{mailto:yy@eyuan.me}{yy@eyuan.me}}%
        \kern 5.0 pt%
        \AND%
        \kern 5.0 pt%
        \mbox{\hrefWithoutArrow{tel:+65-895-234-47}{895 234 47}}%
        \kern 5.0 pt%
        \AND%
        \kern 5.0 pt%
        \mbox{\hrefWithoutArrow{https://www.eyuan.me/}{eyuan.me}}%
        \kern 5.0 pt%
        \AND%
        \kern 5.0 pt%
        \mbox{\hrefWithoutArrow{https://linkedin.com/in/yyuanl}{linkedin.com/in/yyuanl}}%
        \kern 5.0 pt%
        \AND%
        \kern 5.0 pt%
        \mbox{\hrefWithoutArrow{https://github.com/yuann3}{github.com/yuann3}}%
    \end{header}

    \vspace{5 pt - 0.3 cm}


    \section{Summary}
        \begin{onecolentry}
            Resourceful IT graduate fueled by Rust, Go, and backend systems. Seeking a junior backend engineer role to own high-impact APIs and microservices.
        \end{onecolentry}

    \section{Education}
        \begin{twocolentry}{
            Jan 2024 – Sep 2025
        }
        \textbf{The University of Newcastle}, Bachelor in Information Technology\end{twocolentry}
        \vspace{0.10 cm}
        \begin{onecolentry}
            \begin{highlights}
                \item \textbf{High Distinction:} Object-Oriented Programming
            \end{highlights}
            \begin{highlights}
                \item \textbf{Distinction:} Data Structures \& Algorithms, Advanced Database
            \end{highlights}
        \end{onecolentry}

        \vspace{0.20 cm}
        \begin{twocolentry}{
                Jan 2023 – Nov 2023
            }
            \textbf{PSB Academy, Singapore}, Diploma in InfoComm Technology\end{twocolentry}


        \section{Skills \& Technologies}
        
        \subsection*{Technical Skills}
        \begin{onecolentry}
            \textbf{Programming Languages:} Go, Rust, Python, C/C++, Java, SQL, JavaScript (Node.js), C\# 
        \end{onecolentry}
        
        \vspace{0.1 cm}
        
        \begin{onecolentry}
            \textbf{Database \& Storage:} MySQL, SQLite, MSSQL, MongoDB, ChromaDB, Redis
        \end{onecolentry}
        
        \vspace{0.1 cm}
        
        \begin{onecolentry}
            \textbf{Frameworks \& Library:} Flask, .NET MAUI, React.js, Node.js, Raylib
        \end{onecolentry}
        
        \vspace{0.1 cm}
        
        \begin{onecolentry}
            \textbf{Platforms \& OS:} Linux (Debian, Arch, Ubuntu), Unix, Windows, MacOS
        \end{onecolentry}
        
        \vspace{0.1 cm}
        
        \begin{onecolentry}
            \textbf{Architecture \& Performance:} System Design, Distributed Systems, API Structuring; Performance Optimization, Debugging, Unit Testing
        \end{onecolentry}
        
        \vspace{0.1 cm}
        
        \begin{onecolentry}
            \textbf{DevOps \& Tools:} Docker, Git, Jira, Tmux, Neovim, GDB, Makefiles
        \end{onecolentry}
        
        \vspace{0.1 cm}
        
        \begin{onecolentry}
            \textbf{AI \& ML:} Retrieval-Augmented Generation, LLM Workflows, Document Processing, Machine Learning, LLM Fine-Tuning
        \end{onecolentry}
        
        \vspace{0.1 cm}
        
        \begin{onecolentry}
            \textbf{Concepts \& Practices:} System Design, Distributed Systems, API Architecture, Performance Optimization, Debugging, TDD, CI/CD, Unit Testing
        \end{onecolentry}

        \section{Experience}        
                \begin{twocolentry}{
                    Feb 2024 – Present
                }
                    \textbf{Peer Assisted Study Sessions (PASS) Leader}, The University of Newcastle -- Singapore \end{twocolentry}
                \vspace{0.10 cm}
                \begin{onecolentry}
                    \begin{highlights}
                        \item Mentored a group of 10 peers in Data Structures and OOP using C++ and Java, designing custom practice problems and annotated code snippets to reinforce core concepts.
                        \item Delivered 10+ technical mini-lectures on algorithm design and coding practices such as recursion and sorting, using live code walkthroughs in Java and DSA to boost comprehension.
                    \end{highlights}
                \end{onecolentry}
                
                \vspace{0.2 cm}
                
                \vspace{0.2 cm}
                
                \begin{twocolentry}{
                    Sep 2024 – March 2024
                }
                    \textbf{Cadet, Pisciner}, Singapore University of Technology and Design (SUTD), École 42 Programme -- Singapore\end{twocolentry}
                \vspace{0.10 cm}
                \begin{onecolentry}
                    \begin{highlights}
                        \item Completed 16 low-level system projects in C, including push\_swap, libft, and pipex, mastering memory management, pointer arithmetic, and bash scripting.
                        \item Collaborated in a 150\-member cohort using Git for version control and peer code reviews, while debugging memory leaks and writing unit tests in projects like getnextline and ft\_printf.
                    \end{highlights}
                \end{onecolentry}
        
                \vspace{0.2 cm}
    
        \section{Projects}
        
        \begin{twocolentry}{
            \href{https://github.com/yuann3/Ruskey/}{Personal Project}
        }
            \textbf{Ruskey: Monkey Programming Language Interpreter}\end{twocolentry}

        \vspace{0.10 cm}
        \begin{onecolentry}
            \begin{highlights}
                \item Built a full-featured interpreter for the Monkey language in Rust, including a lexer, parser, AST, and evaluator, to demonstrate parsing and language runtime implementation.
                \item Integrated all interpreter components — lexer, parser, AST, and evaluator — with full test coverage to validate correctness and improve debugging efficiency.        
                \item Supported language features including booleans, integers, closures, and first-class functions to reflect real-world scripting capabilities.
                \item Tools Used: Rust, Test-Driven Development, Abstract Syntax Trees, Recursive Descent Parsing
            \end{highlights}
        \end{onecolentry}

        \vspace{0.2 cm} 

        \begin{twocolentry}{
            \href{https://github.com/yuann3/Pew/}{Personal Project}
        }
            \textbf{Rego: Redis DB implementation in Go}\end{twocolentry}

        \vspace{0.10 cm}
        \begin{onecolentry}
            \begin{highlights}
                \item Implemented core Redis commands and WAIT replication in Go, passing full Codecrafters tests
                \item Applied slice optimizations to minimize GC overhead and improved throughput by 30%
                \item Tools Used: Golang, Database
            \end{highlights}
        \end{onecolentry}

        \vspace{0.2 cm} 

        \begin{twocolentry}{
            \href{https://github.com/yuann3/Pew/}{Personal Project}
        }
            \textbf{Pew: Lightweight CLI for Code Dumping}\end{twocolentry}

        \vspace{0.10 cm}
        \begin{onecolentry}
            \begin{highlights}
                \item Built a CLI tool in Golang to package entire codebases into a Markdown file for streamlined input into LLM pipelines and documentation workflows.
                \item Implemented file parsing, Gitignore-style pattern matching, and syntax-highlighted output with tree-style directory visualization.
                \item Tools Used: Golang, CLI Development, File I/O
            \end{highlights}
        \end{onecolentry}


        \vspace{0.2 cm}

        \begin{twocolentry}{
            \href{https://github.com/yuann3/http-rust/}{Personal Project}
        }
            \textbf{Rust HTTP Server}\end{twocolentry}

        \vspace{0.10 cm}
        \begin{onecolentry}
            \begin{highlights}
                \item Developed a multithreaded HTTP/1.1 server in Rust supporting GET/POST requests, file uploads, and gzip compression, with optimized request handling and concurrency.
                \item Implemented a User-Agent echo endpoint to assist in request debugging and improved response throughput through thread pooling and efficient I/O operations.
                \item Tools Used: Rust, HTTP/1.1, Multithreading
            \end{highlights}
        \end{onecolentry}


        \vspace{0.2 cm}
        
        \begin{twocolentry}{
            \href{https://github.com/yuann3/Hiraku-RAG}{The University of Newcastle}
        }
            \textbf{Hiraku: AI-Powered Smart Learning Companion}\end{twocolentry}

        \vspace{0.10 cm}
        \begin{onecolentry}
            \begin{highlights}
                \item Led backend architecture of an AI assistant using Python (Flask) and SQLite, integrating RAG pipelines for intelligent PDF/TXT document parsing and retrieval.
                \item Built a secure REST API with JWT-based access control and optimized document vector indexing via ChromaDB for fast semantic retrieval.
                \item Tool Used: JavaScript/TypeScript (React 19, Next.js), Tailwind CSS, Python (Flask REST API, JWT), Database (SQLite)
            \end{highlights}
        \end{onecolentry}


        \vspace{0.2 cm}

        \begin{twocolentry}{
            \href{https://github.com/yuann3/RayRust}{Personal Project}
        }
            \textbf{Ray Tracer in Rust}\end{twocolentry}

        \vspace{0.10 cm}
        \begin{onecolentry}
            \begin{highlights}
                \item A ray tracer was developed from scratch in Rust. It features robust 3D vector mathematics, comprehensive operator overloading, and thorough unit testing. An efficient camera system was designed, incorporating viewport calculations and ray generation capabilities. Rust's type system was utilized to ensure strong compile-time guarantees and memory safety
                \item Tools Used: Rust, Linear Algebra, Computer Graphics, PPM Image Format
            \end{highlights}
        \end{onecolentry}

        \vspace{0.2 cm}

        \begin{twocolentry}{
            \href{https://github.com/yuann3/Ylib}{École 42}
        }
            \textbf{Ylib: C Standard Library Rewrite}\end{twocolentry}

        \vspace{0.10 cm}
        \begin{onecolentry}
            \begin{highlights}
                \item Ylib is a modern rewrite of the C Standard Library that aims to enhance performance and maintainability while adhering to the ANSI C standard. It includes core C library components such as string manipulation, memory management, I/O operations, and math functions.
                \item Tools Used: C, Makefile, GCC, GDB
            \end{highlights}
        \end{onecolentry}

        \vspace{0.2 cm}

        \begin{twocolentry}{
            \href{https://github.com/yuann3/push_swap}{École 42}
        }
            \textbf{Push Swap: Stack Sorting Algorithm}\end{twocolentry}

        \vspace{0.10 cm}
        \begin{onecolentry}
            \begin{highlights}
                \item Implemented a stack-based integer sorting algorithm in C, minimizing operations through greedy and divide-and-conquer strategies, achieving $\mathcal{O}(d(n + k))$ efficiency
                \item Tools Used: C, Makefile, GCC, GDB
            \end{highlights}
        \end{onecolentry}


    

\end{document}